\apendice{Plan de Proyecto Software}

\section{Introducción}

\section{Planificación temporal}

La planificación temporal se ha realizado adaptando la metodología \textit{Scrum} a un proyecto educativo, con los cambios que esto conlleva.

\begin{itemize}
	\item El desarrollo se ha basado en iteraciones o \textit{sprints} de una semana de duración aproximadamente.
	\item  Cada uno de los \textit{sprints} contiene las tareas o \textit{issues} que se realizaron esa semana. 
	\item Cada tarea tiene asociado un coste, que simboliza su dificultad en cuanto al esfuerzo que se estima invertir en ella.
	\item En caso de que la estimación del coste resultara inexacta al realizar el \textit{issue}, este se modificó para reflejar el esfuerzo real empleado.
	\item Al finalizar cada \textit{sprint} se realizaba una reunión de revisión con los tutores donde se exponían los progresos realizados y se planificaba el siguiente \textit{sprint}.
\end{itemize}

\subsection{Sprint 1}

Fecha: 19/12/1018 - 23/12/2018

El primer \textit{sprint} consistió en realizar una exploración bibliográfica inicial sobre el estado del arte. 

\begin{table}[h]
	 \begin{tabularx}{\linewidth}{|Xrr|}
	 	\hline \textbf{\textit{Issue}} & \textbf{Estimado} & \textbf{Final}\\
	 	\hline Crear y configurar repositorio & 2 & 2 \\
	 	\hline Exploración bibliográfica inicial & 13 & 13 \\
	 	\hline
	 \end{tabularx}
\end{table}

\subsection{Sprint 2}

Fecha: 23/12/2018 - 29/12/2018

Se continuó la exploración bibliográfica inicial, centrándose en artículos especialmente interesantes encontrados hasta el momento y se comenzó la exploración bibliográfica sobre otros métodos aplicables al problema. 

\begin{table}[h]
	\begin{tabularx}{\textwidth}{|Xrr|}
		\hline \textbf{\textit{Issue}} & \textbf{Estimado} & \textbf{Final}\\
		\hline Continuación de la exploración bibliográfica inicial & 8 & 8 \\
		\hline Exploración bibliográfica sobre otros métodos aplicables al problema & 8 & 8 \\
		\hline Lectura de "Automated Epileptic Seizure Detection Methods: A Review Study" & 8 & 8 \\
		\hline Instalar y configurar cliente VPN & 2 & 2 \\
		\hline
	\end{tabularx}
\end{table}

\subsection{Sprint 3}

Fecha: 29/12/2018 - 11/01/2019

Se inició la documentación y se empezó a trabajar en la visualización de los datos en bruto y de algunos datos estadísticos.

\begin{table}[h]
	\begin{tabularx}{\textwidth}{|Xrr|}
		\hline \textbf{\textit{Issue}} & \textbf{Estimado} & \textbf{Final}\\
		\hline Iniciar documentación &  &  \\
		\hline Instalar entorno y librerías de Python & 5 & 5 \\
		\hline Aprender a usar librerías & 8 & 8 \\
		\hline Procesar y mostrar datos & 8 & 8 \\
		\hline
	\end{tabularx}
\end{table}

\subsection{Sprint 4}

Fecha: 11/01/2019 - 18/01/2019

Se configuró el acceso al computador del departamento para probar técnicas de reducción de la dimensionalidad de los datos y algunas opciones básicas de filtrado y suavizado de la señal. 

\begin{table}[h]
	\begin{tabularx}{\textwidth}{|Xrr|}
		\hline \textbf{\textit{Issue}} & \textbf{Estimado} & \textbf{Final}\\
		\hline Configurar acceso a gamma & 5 & 5 \\
		\hline Probar opciones filtrado y suavizado & 8 & 8 \\
		\hline Probar otras formas de proyección de datos & 8 & 21 \\
		\hline
	\end{tabularx}
\end{table}

\subsection{Sprint 5}

Fecha: 18/01/2019 - 25/01/2019

Se hicieron cambios en el preprocesado, se probaron otras formas de filtrado de la señal y se estudiaron los puntos clave de las proyecciones del \textit{sprint} anterior. 

\begin{table}[h]
	\begin{tabularx}{\textwidth}{|Xrr|}
		\hline \textbf{\textit{Issue}} & \textbf{Estimado} & \textbf{Final}\\
		\hline Leer apuntes de minería de datos & 8 & 8 \\
		\hline Modificar preprocesado & 3 & 3 \\
		\hline Representar señales en torno a la crisis epiléptica & 5 & 5 \\
		\hline Probar formas de filtrado de la señal & 5 & 5 \\
		\hline Estudiar los puntos clave de las proyecciones & 5 & 8 \\
		\hline
	\end{tabularx}
\end{table}

\subsection{Sprint 6}

Fecha: 25/01/2019 - 31/02/2019

Se centraron las pruebas en las proyecciones con mejor rendimiento, concretamente en MDS, y se iniciaron la documentación de la planificación temporal y el cuaderno de investigación. 

\begin{table}[h]
	\begin{tabularx}{\textwidth}{|Xrr|}
		\hline \textbf{\textit{Issue}} & \textbf{Estimado} & \textbf{Final}\\
		\hline Cambiar a proyecciones con mejor rendimiento & 13 & 13 \\
		\hline Pasar cálculos estadísticos a funciones & 5 & 5 \\
		\hline Documentar 5 primeros Sprints en el Plan de Proyecto & 8 & 8 \\
		\hline Documentar investigación en overleaf & 5 & 5 \\
		\hline
	\end{tabularx}
\end{table}

\subsection{Sprint 7}

Fecha: 31/02/2019 - 07/02/2019

Se codificaron las transformaciones generadas en los \textit{sprints} anteriores (normalización, filtros y estadísticas) como transformadores de sklearn.

\begin{table}[h]
	\begin{tabularx}{\textwidth}{|Xrr|}
		\hline \textbf{\textit{Issue}} & \textbf{Estimado} & \textbf{Final}\\
		\hline Aprender sobre la clase sklearn.base.TransformerMixin & 3 & 3 \\
		\hline Generar transformadores para las funciones usadas & 21 & 13 \\
		\hline
	\end{tabularx}
\end{table}

\subsection{Sprint 8}

Fecha: 07/02/2019 - 14/02/2019

Se exploraron otras formas de proyección y se realizó una primera aproximación de clasificación mediante Random Forest y detección de anomalías One-class. 

\begin{table}[h]
	\begin{tabularx}{\textwidth}{|Xrr|}
		\hline \textbf{\textit{Issue}} & \textbf{Estimado} & \textbf{Final}\\
		\hline Probar Kernel PCA & 5 & 5 \\
		\hline Acotar ataque a partir del aspecto de la señal y la salida de las proyecciones (MDS) & 3 & 5 \\
		\hline Probar MDS con el ataque reetiquetado & 8 & 8 \\
		\hline Probar clasificador Random Forest & 8 & 8 \\
		\hline Aplicar detección de anomalías one-class & 13 & 13 \\
		\hline
	\end{tabularx}
\end{table}


\subsection{Sprint 9}

Fecha: 14/02/2019 - 21/02/2019

Se planteó la evaluación de los clasificadores mediante el área bajo la curva ROC y se terminaron de documentar las proyecciones en el cuaderno de investigación. 

\begin{table}[h]
	\begin{tabularx}{\textwidth}{|Xrr|}
		\hline \textbf{\textit{Issue}} & \textbf{Estimado} & \textbf{Final}\\
		\hline Valorar los resultados de Random Forest mediante el área bajo la curva & 3 & 3 \\
		\hline Incluir las proyecciones en la documentación & 5 & 5 \\
		\hline Preparar la visualización de las proyecciones para la documentación & 5 & 8 \\
		\hline
	\end{tabularx}
\end{table}

\subsection{Sprint 10}

Fecha: 21/02/2019 - 28/02/2019

Una parte se invirtió en aprender sobre clasificación de conjuntos de datos desequilibrados mediante ensembles y por otro lado se realizó una exploración de ventanas para la aplicación de los datos al clasificador Random Forest. 

\begin{table}[h]
	\begin{tabularx}{\textwidth}{|Xrr|}
		\hline \textbf{\textit{Issue}} & \textbf{Estimado} & \textbf{Final}\\
		\hline Lectura de aprendizaje sobre datos desequilibrados & - & - \\
		\hline Aplicar Random Forest a datos estadísticos con distintas ventanas & 8 & 8 \\
		\hline
	\end{tabularx}
\end{table}

\subsection{Sprint 11}

Fecha: 28/02/2019 - 07/03/2019

Se continuó con la lectura sobre desequilibrados y se inició el aprendizaje sobre la librería tsfresh para extracción de características en series temporales.  

\begin{table}[h]
	\begin{tabularx}{\textwidth}{|Xrr|}
		\hline \textbf{\textit{Issue}} & \textbf{Estimado} & \textbf{Final}\\
		\hline Continuar con la lectura sobre uso de ensembles para conjuntos desequilibrados & 5 & 5 \\
		\hline Extracción de características en series temporales & 8 & 8 \\
		\hline
	\end{tabularx}
\end{table}

\subsection{Sprint 12}

Fecha: 07/03/2019 - 14/03/2019

Se trataron de aplicar los resultados de la extracción de características de series temporales al clasificador Random Forest. 

\begin{table}[h]
	\begin{tabularx}{\textwidth}{|Xrr|}
		\hline \textbf{\textit{Issue}} & \textbf{Estimado} & \textbf{Final}\\
		\hline Random Forest con características de series temporales & 5 & 5 \\
		\hline Continuar extracción de características en series temporales & 13 & 13 \\
		\hline
	\end{tabularx}
\end{table}

\subsection{Sprint 13}

Fecha: 14/03/2019 - 21/03/2019

Principalmente se exploraron formas de filtrar y combinar las mejores características de series temporales para ser aplicadas al clasificador. 

\begin{table}[h]
	\begin{tabularx}{\textwidth}{|Xrr|}
		\hline \textbf{\textit{Issue}} & \textbf{Estimado} & \textbf{Final}\\
		\hline Filtrado de características & 13 & 5 \\
		\hline Aplicar Random Forest a combinaciones de las mejores características & 5 & 5 \\
		\hline Documentación de sprints pasados y actualización del cuaderno de investigación & 5 & 5 \\
		\hline
	\end{tabularx}
\end{table}

\subsection{Sprint 14}

Fecha: 21/03/2019 - 28/03/2019

Se planteó un filtrado de características mediante un algoritmo genético usando el framework deap de python y se realizó una investigación inicial de técnicas para la implementación de servidores de streaming. 

\begin{table}[h]
	\begin{tabularx}{\textwidth}{|Xrr|}
		\hline \textbf{\textit{Issue}} & \textbf{Estimado} & \textbf{Final}\\
		\hline Investigar técnicas para implementar servidores de streaming & 8 & 8 \\
		\hline Algoritmo genético para la selección de características & 13 & 13 \\
		\hline Avanzar con la documentación en el cuaderno de investigación & 5 & 13 \\
		\hline
	\end{tabularx}
\end{table}

\subsection{Sprint 15}

Fecha: 28/03/2019 - 04/04/2019

Se mejoró el algoritmo genético, se finalizó su ejecución con la ayuda de tmux y se documentaron los resultados. Además, se inició el diseño de los requisitos y los casos de uso de la aplicación y se plantearon los primeros prototipos. 

\begin{table}[h]
	\begin{tabularx}{\textwidth}{|Xrr|}
		\hline \textbf{\textit{Issue}} & \textbf{Estimado} & \textbf{Final}\\
		\hline Generar prototipo de la pantalla de visualización de datos & 5 & 5 \\
		\hline Aprender sobre tmux para la ejecución del genético & 3 & 3 \\
		\hline Mejorar algoritmo genético y documentar resultados & 8 & 8 \\
		\hline Plantear primeras cuestiones de diseño de la app & 5 & 5 \\
		\hline
	\end{tabularx}
\end{table}

\subsection{Sprint 16}

Fecha: 04/04/2019 - 11/04/2019

Se ultimaron los detalles del cuaderno de investigación con la documentación generada hasta el momento, se finalizaron los prototipos y se documentó la parte de diseño y de las técnicas. Además, se instaló Android Studio para su uso en sprints posteriores. 


\begin{table}[h]
	\begin{tabularx}{\textwidth}{|Xrr|}
		\hline \textbf{\textit{Issue}} & \textbf{Estimado} & \textbf{Final}\\
		\hline Ultimar detalles del cuaderno e trabajo & - & - \\
		\hline Finalizar y documentar prototipos & 5 & 5 \\
		\hline Avanzar en la documentación temporal, de diseño y de las técnicas & 13 & 13 \\
		\hline Instalar Android Studio & 2 & 2 \\
		\hline
	\end{tabularx}
\end{table}

\section{Estudio de viabilidad}

\subsection{Viabilidad económica}

\subsection{Viabilidad legal}


