\apendice{Plan de Proyecto Software}

\section{Introducción}

En este apartado se va exponer la planificación temporal del proyecto, indicando qué tareas y cuándo se realizaron. Además, se presenta un análisis de la viabilidad legal y económica del proyecto. 

\section{Planificación temporal}

La planificación temporal se ha realizado adaptando la metodología \textit{Scrum} a un proyecto educativo, con los cambios que esto conlleva.

\begin{itemize}
	\item El desarrollo se ha basado en iteraciones o \textit{sprints} de una semana de duración aproximadamente.
	\item  Cada uno de los \textit{sprints} contiene las tareas o \textit{issues} que se realizaron esa semana. 
	\item Cada tarea tiene asociado un coste, que simboliza su dificultad en cuanto al esfuerzo que se estima invertir en ella.
	\item En caso de que la estimación del coste resultara inexacta al realizar el \textit{issue}, este se modificó para reflejar el esfuerzo real empleado.
	\item Al finalizar cada \textit{sprint} se realizaba una reunión de revisión con los tutores donde se exponían los progresos realizados y se planificaba el siguiente \textit{sprint}.
\end{itemize}

\subsection{Sprint 1}

Fecha: 19/12/1018 - 23/12/2018

El primer \textit{sprint} consistió en realizar una exploración bibliográfica inicial sobre el estado del arte. 

\begin{table}[H]
	 \begin{tabularx}{\linewidth}{X r r}
	 	\toprule \textbf{\textit{Issue}} & \textbf{Estimado} & \textbf{Final}\\
	 	\toprule
	 	Crear y configurar repositorio & 2 & 2 \\
	 	Exploración bibliográfica inicial & 13 & 13 \\
	 	\bottomrule
	 \end{tabularx}
	 \caption{Tareas del sprint 1}
\end{table}

\subsection{Sprint 2}

Fecha: 23/12/2018 - 29/12/2018

Se continuó la exploración bibliográfica inicial, centrándose en artículos especialmente interesantes encontrados hasta el momento y se comenzó la exploración bibliográfica sobre otros métodos aplicables al problema. 

\begin{table}[H]
	\begin{tabularx}{\textwidth}{Xrr}
		\toprule \textbf{\textit{Issue}} & \textbf{Estimado} & \textbf{Final}\\
		\toprule
		Continuación de la exploración bibliográfica inicial & 8 & 8 \\
		Exploración bibliográfica sobre otros métodos aplicables al problema & 8 & 8 \\
		Lectura de <<Automated Epileptic Seizure Detection Methods: A Review Study>> & 8 & 8 \\
		Instalar y configurar cliente VPN & 2 & 2 \\
		\bottomrule
	\end{tabularx}
	\caption{Tareas del sprint 2}
\end{table}

\subsection{Sprint 3}

Fecha: 29/12/2018 - 11/01/2019

Se inició la documentación y se empezó a trabajar en la visualización de los datos en bruto y de algunos datos estadísticos.

\begin{table}[H]
	\begin{tabularx}{\textwidth}{Xrr}
		\toprule \textbf{\textit{Issue}} & \textbf{Estimado} & \textbf{Final}\\
		\toprule 
		Iniciar documentación de los sprints & - & 5 \\
		Instalar entorno y librerías de Python & 5 & 5 \\
		Aprender a usar librerías & 8 & 8 \\
		Procesar y mostrar datos & 8 & 8 \\
		\bottomrule
	\end{tabularx}
	\caption{Tareas del sprint 3}
\end{table}

\subsection{Sprint 4}

Fecha: 11/01/2019 - 18/01/2019

Se configuró el acceso al equipo de cómputo del grupo de investigación para probar técnicas de reducción de la dimensionalidad de los datos y algunas opciones básicas de filtrado y suavizado de la señal. 

\begin{table}[H]
	\begin{tabularx}{\textwidth}{Xrr}
		\toprule \textbf{\textit{Issue}} & \textbf{Estimado} & \textbf{Final}\\
		\toprule
		Configurar acceso a gamma & 5 & 5 \\
		Probar opciones filtrado y suavizado & 8 & 8 \\
		Probar otras formas de proyección de datos & 8 & 21 \\
		\bottomrule
	\end{tabularx}
	\caption{Tareas del sprint 4}
\end{table}

\subsection{Sprint 5}

Fecha: 18/01/2019 - 25/01/2019

Se hicieron cambios en el preprocesado, se probaron otras formas de filtrado de la señal y se estudiaron los puntos clave de las proyecciones del \textit{sprint} anterior. 

\begin{table}[H]
	\begin{tabularx}{\textwidth}{Xrr}
		\toprule \textbf{\textit{Issue}} & \textbf{Estimado} & \textbf{Final}\\
		\toprule
		Leer apuntes de minería de datos & 8 & 8 \\
		Modificar preprocesado & 3 & 3 \\
		Representar señales en torno a la crisis epiléptica & 5 & 5 \\
		Probar formas de filtrado de la señal & 5 & 5 \\
		Estudiar los puntos clave de las proyecciones & 5 & 8 \\
		\bottomrule
	\end{tabularx}
	\caption{Tareas del sprint 5}
\end{table}

\subsection{Sprint 6}

Fecha: 25/01/2019 - 31/01/2019

Se centraron las pruebas en las proyecciones con mejor rendimiento, concretamente en MDS~\cite{MDS}, y se iniciaron la documentación de la planificación temporal y el cuaderno de investigación. 

\begin{table}[H]
	\begin{tabularx}{\textwidth}{Xrr}
		\toprule \textbf{\textit{Issue}} & \textbf{Estimado} & \textbf{Final}\\
		\toprule
		Cambiar a proyecciones con mejor rendimiento & 13 & 13 \\
		Pasar cálculos estadísticos a funciones & 5 & 5 \\
		Documentar 5 primeros Sprints en el Plan de Proyecto & 8 & 8 \\
		Documentar investigación en Overleaf~\cite{overleaf} & 5 & 5 \\
		\bottomrule
	\end{tabularx}
	\caption{Tareas del sprint 6}
\end{table}

\subsection{Sprint 7}

Fecha: 31/01/2019 - 07/02/2019

Se codificaron las transformaciones generadas en los \textit{sprints} anteriores (normalización, filtros y estadísticas) como transformadores de Sklearn~\cite{scikit-learn}.

\begin{table}[H]
	\begin{tabularx}{\textwidth}{Xrr}
		\toprule \textbf{\textit{Issue}} & \textbf{Estimado} & \textbf{Final}\\
		\toprule 
		Aprender sobre la clase TransformerMixin del paquete sklearn.base~\cite{TransformerMixin} & 3 & 3 \\
		Generar transformadores para las funciones usadas & 21 & 13 \\
		\bottomrule
	\end{tabularx}
	\caption{Tareas del sprint 7}
\end{table}

\subsection{Sprint 8}

Fecha: 07/02/2019 - 14/02/2019

Se exploraron otras formas de proyección y se realizó una primera aproximación de clasificación mediante Random Forest~\cite{RandomForest} y detección de anomalías One-class~\cite{OneClass}. 

\begin{table}[H]
	\begin{tabularx}{\textwidth}{Xrr}
		\toprule \textbf{\textit{Issue}} & \textbf{Estimado} & \textbf{Final}\\
		\toprule 
		Probar Kernel PCA & 5 & 5 \\
		Acotar ataque a partir del aspecto de la señal y la salida de las proyecciones (MDS) & 3 & 5 \\
		Probar MDS con el ataque reetiquetado & 8 & 8 \\
		Probar clasificador Random Forest & 8 & 8 \\
		Aplicar detección de anomalías one-class & 13 & 13 \\
		\bottomrule
	\end{tabularx}
	\caption{Tareas del sprint 8}
\end{table}


\subsection{Sprint 9}

Fecha: 14/02/2019 - 21/02/2019

Se planteó la evaluación de los clasificadores mediante el área bajo la curva ROC y se terminaron de documentar las proyecciones en el cuaderno de investigación. 

\begin{table}[H]
	\begin{tabularx}{\textwidth}{Xrr}
		\toprule \textbf{\textit{Issue}} & \textbf{Estimado} & \textbf{Final}\\
		\toprule 
		Valorar los resultados de Random Forest mediante el área bajo la curva & 3 & 3 \\
		Incluir las proyecciones en la documentación & 5 & 5 \\
		Preparar la visualización de las proyecciones para la documentación & 5 & 8 \\
		\bottomrule
	\end{tabularx}
	\caption{Tareas del sprint 9}
\end{table}

\subsection{Sprint 10}

Fecha: 21/02/2019 - 28/02/2019

Una parte se invirtió en aprender sobre clasificación de conjuntos de datos desequilibrados mediante ensembles y por otro lado se realizó una exploración de ventanas para la aplicación de los datos al clasificador Random Forest. 

\begin{table}[H]
	\begin{tabularx}{\textwidth}{Xrr}
		\toprule \textbf{\textit{Issue}} & \textbf{Estimado} & \textbf{Final}\\
		\toprule 
		Lectura de aprendizaje sobre datos desequilibrados & - & 8 \\
		Aplicar Random Forest a datos estadísticos con distintas ventanas & 8 & 8 \\
		\bottomrule
	\end{tabularx}
	\caption{Tareas del sprint 10}
\end{table}

\subsection{Sprint 11}

Fecha: 28/02/2019 - 07/03/2019

Se continuó con la lectura sobre desequilibrados y se inició el aprendizaje sobre la librería tsfresh~\cite{tsfresh} para extracción de características en series temporales.  

\begin{table}[H]
	\begin{tabularx}{\textwidth}{Xrr}
		\toprule \textbf{\textit{Issue}} & \textbf{Estimado} & \textbf{Final}\\
		\toprule
		Continuar con la lectura sobre uso de ensembles para conjuntos desequilibrados & 5 & 5 \\
		Extracción de características en series temporales & 8 & 8 \\
		\bottomrule
	\end{tabularx}
	\caption{Tareas del sprint 11}
\end{table}

\subsection{Sprint 12}

Fecha: 07/03/2019 - 14/03/2019

Se trataron de aplicar los resultados de la extracción de características de series temporales al clasificador Random Forest. 

\begin{table}[H]
	\begin{tabularx}{\textwidth}{Xrr}
		\toprule \textbf{\textit{Issue}} & \textbf{Estimado} & \textbf{Final}\\
		\toprule
		Random Forest con características de series temporales & 5 & 5 \\
		Continuar extracción de características en series temporales & 13 & 13 \\
		\bottomrule
	\end{tabularx}
	\caption{Tareas del sprint 12}
\end{table}

\subsection{Sprint 13}

Fecha: 14/03/2019 - 21/03/2019

Principalmente se exploraron formas de filtrar y combinar las mejores características de series temporales para ser aplicadas al clasificador. 

\begin{table}[H]
	\begin{tabularx}{\textwidth}{Xrr}
		\toprule \textbf{\textit{Issue}} & \textbf{Estimado} & \textbf{Final}\\
		\toprule 
		Filtrado de características & 13 & 5 \\
		Aplicar Random Forest a combinaciones de las mejores características & 5 & 5 \\
		Documentación de sprints pasados y actualización del cuaderno de investigación & 5 & 5 \\
		\bottomrule
	\end{tabularx}
	\caption{Tareas del sprint 13}
\end{table}

\subsection{Sprint 14}

Fecha: 21/03/2019 - 28/03/2019

Se planteó un filtrado de características mediante un algoritmo genético usando el framework DEAP~\cite{deap} de python y se realizó una investigación inicial de técnicas para la implementación de servidores de \textit{streaming}. 

\begin{table}[H]
	\begin{tabularx}{\textwidth}{Xrr}
		\toprule \textbf{\textit{Issue}} & \textbf{Estimado} & \textbf{Final}\\
		\toprule 
		Investigar técnicas para implementar servidores de streaming & 8 & 8 \\
		Algoritmo genético para la selección de características & 13 & 13 \\
		Avanzar con la documentación en el cuaderno de investigación & 5 & 13 \\
		\bottomrule
	\end{tabularx}
	\caption{Tareas del sprint 14}
\end{table}

\subsection{Sprint 15}

Fecha: 28/03/2019 - 04/04/2019

Se mejoró el algoritmo genético, se finalizó su ejecución con la ayuda de tmux~\cite{tmux} y se documentaron los resultados. Además, se inició el diseño de los requisitos y los casos de uso de la aplicación y se plantearon los primeros prototipos. 

\begin{table}[H]
	\begin{tabularx}{\textwidth}{Xrr}
		\toprule \textbf{\textit{Issue}} & \textbf{Estimado} & \textbf{Final}\\
		\toprule 
		Generar prototipo de la pantalla de visualización de datos & 5 & 5 \\
		Aprender sobre tmux para la ejecución del genético & 3 & 3 \\
		Mejorar algoritmo genético y documentar resultados & 8 & 8 \\
		Plantear primeras cuestiones de diseño de la app & 5 & 5 \\
		\bottomrule
	\end{tabularx}
	\caption{Tareas del sprint 15}
\end{table}

\subsection{Sprint 16}

Fecha: 04/04/2019 - 11/04/2019

Se ultimaron los detalles del cuaderno de investigación con la documentación generada hasta el momento, se finalizaron los prototipos y se documentó la parte de diseño y de las técnicas. Además, se instaló Android Studio para su uso en sprints posteriores. 


\begin{table}[H]
	\begin{tabularx}{\textwidth}{Xrr}
		\toprule \textbf{\textit{Issue}} & \textbf{Estimado} & \textbf{Final}\\
		\toprule
		Ultimar detalles del cuaderno e trabajo & - & 8 \\
		Finalizar y documentar prototipos & 5 & 5 \\
		Avanzar en la documentación temporal, de diseño y de las técnicas & 13 & 13 \\
		Instalar Android Studio & 2 & 2 \\
		\bottomrule
	\end{tabularx}
	\caption{Tareas del sprint 16}
\end{table}

\subsection{Sprint 17}

Fecha: 11/04/2019 - 18/04/2019

Se refactorizó el código de los experimentos para incluir el testeo mediante la métrica precision-recall, más adecuada para conjuntos de datos desequilibrados, y se volvieron a ejecutar los filtrados de características. 

\begin{table}[H]
	\begin{tabularx}{\textwidth}{Xrr}
		\toprule \textbf{\textit{Issue}} & \textbf{Estimado} & \textbf{Final}\\
		\toprule
		Refactorizar el código de Random Forest para incluir la métrica precision-recall & 8 & 8 \\
		Volver a ejecutar los filtrados de características para la nueva métrica & 8 & 8 \\
		\bottomrule
	\end{tabularx}
	\caption{Tareas del sprint 17}
\end{table}

\subsection{Sprint 18}

Fecha: 18/04/2019 - 02/05/2019

Se documentaron los resultados de los filtrados de características del sprint anterior y se comenzó la lectura sobre la documentación de Android Studio y la visualización del curso \textit{Android Development for Beginners} de Google.

\begin{table}[H]
	\begin{tabularx}{\textwidth}{Xrr}
		\toprule \textbf{\textit{Issue}} & \textbf{Estimado} & \textbf{Final}\\
		\toprule
		Documentar los resultados de las nuevas ejecuciones en el cuaderno de investigación & 8 & 8 \\
		Aprender a usar Android Studio & 13 & 21 \\
		\bottomrule
	\end{tabularx}
	\caption{Tareas del sprint 18}
\end{table}

\subsection{Sprint 19}

Fecha: 02/05/2019 - 09/05/2019

Se terminó el comportamiento de la pantalla de autenticación, se generó el clasificador obtenido con el mejor conjunto de características encontrado y se continuó con la documentación de la planificación temporal. 

\begin{table}[H]
	\begin{tabularx}{\textwidth}{Xrr}
		\toprule \textbf{\textit{Issue}} & \textbf{Estimado} & \textbf{Final}\\
		\toprule
		Terminar el comportamiento de la pantalla de login & 3 & 3 \\
		Extraer características deseadas con tsfresh y generar clasificador & 5 & 5 \\
		Documentación de la memoria & 3 & 3 \\
		\bottomrule
	\end{tabularx}
	\caption{Tareas del sprint 19}
\end{table}

\subsection{Sprint 20}

Fecha: 02/05/2019 - 16/05/2019

Se crearon las pantallas principales de la aplicación de Android.

\begin{table}[H]
	\begin{tabularx}{\textwidth}{Xrr}
		\toprule \textbf{\textit{Issue}} & \textbf{Estimado} & \textbf{Final}\\
		\toprule
		Crear la pantalla de visualización de camas & 8 & 13 \\
		Crear pantalla de visualización de datos & 13 & 13 \\
		Terminar pantallas de gestión de Usuarios & 5 & 5 \\
		\bottomrule
	\end{tabularx}
	\caption{Tareas del sprint 20}
\end{table}

\subsection{Sprint 21}

Se terminó el comportamiento general de la aplicación de Android, se solucionaron algunos bugs y se adecuaron las interfaces de usuario a la guía de estilos empleada. 

Fecha: 16/05/2019 - 23/05/2019

\begin{table}[H]
	\begin{tabularx}{\textwidth}{Xrr}
		\toprule \textbf{\textit{Issue}} & \textbf{Estimado} & \textbf{Final}\\
		\toprule
		Terminar comportamiento de la pantalla de visualización de datos & 5 & 5 \\
		Pantallas de gestión de camas & 8 & 8 \\
		Modificar estructura de las pantallas de gestión de usuarios & 5 & 5 \\
		Bug en la visualización de los datos & 3 & 3 \\
		Modificaciones menores de las interfaces para adecuarse a la guía de estilos & 5 & 5 \\
		Crear menú para el rol de usuario & 8 & 13 \\
		\bottomrule
	\end{tabularx}
	\caption{Tareas del sprint 21}
\end{table}

\subsection{Sprint 22}

Fecha: 23/05/2019 - 30/05/2019

Se codificó el comportamiento de la aplicación ante pérdidas de conexión y ante pérdidas de la sesión por autenticación con el mismo usuario en otro dispositivo. 

\begin{table}[H]
	\begin{tabularx}{\textwidth}{Xrr}
		\toprule \textbf{\textit{Issue}} & \textbf{Estimado} & \textbf{Final}\\
		\toprule
		Comprobar si la sesión ha caducado y redirigir & 3 & 3 \\
		Controlar la pérdida de conexión a internet & 5 & 5 \\
		\bottomrule
	\end{tabularx}
	\caption{Tareas del sprint 22}
\end{table}

\subsection{Sprint 23}

Fecha: 30/05/2019 - 07/06/2019

Se solucionaron varios bugs de la aplicación. 

\begin{table}[H]
	\begin{tabularx}{\textwidth}{Xrr}
		\toprule \textbf{\textit{Issue}} & \textbf{Estimado} & \textbf{Final}\\
		\toprule
		Bug en la pantalla de gráficas & 5 & 5 \\
		Bug en la pantalla de administración cuando la lista supera la longitud de la pantalla & 5 & 5 \\
		Cambiar cambio de contraseña de usuarios desde el administrador & 2 & 2 \\
		\bottomrule
	\end{tabularx}
	\caption{Tareas del sprint 23}
\end{table}

\subsection{Sprint 24}

Fecha: 07/06/2019 - 13/06/2019

Se comenzó a redactar la memoria y se eliminaron algunos bugs de la aplicación. 

\begin{table}[H]
	\begin{tabularx}{\textwidth}{Xrr}
		\toprule \textbf{\textit{Issue}} & \textbf{Estimado} & \textbf{Final}\\
		\toprule
		Comenzar documentación de la memoria & 13 & 13 \\
		Bug de las gráficas al refrescar & 8 & 8 \\
		\bottomrule
	\end{tabularx}
	\caption{Tareas del sprint 24}
\end{table}

\subsection{Sprint 25}

Fecha: 13/06/2019 - 19/06/2019

Se terminó la primera versión de la memoria, se generó el archivo apk definitivo y se probó en distintos dispositivos compatibles. 

\begin{table}[H]
	\begin{tabularx}{\textwidth}{Xrr}
		\toprule \textbf{\textit{Issue}} & \textbf{Estimado} & \textbf{Final}\\
		\toprule
		Eliminar filas huérfanas y viudas & - & 1 \\
		Terminar la primera versión de la memoria & 13 & 21 \\
		Generar apk y probarla en varios dispositivos & 3 & 3 \\
		Eliminar errores al instalar en otros dispositivos & 5 & 5 \\
		\bottomrule
	\end{tabularx}
	\caption{Tareas del sprint 25}
\end{table}

\subsection{Sprint 26}

Fecha: 19/06/2019 - 27/06/2019

Se añadió una pantalla <<about>> a la aplicación, se comentó y documentó el código usando javadoc y se terminó la documentación de los anexos. 

\begin{table}[H]
	\begin{tabularx}{\textwidth}{Xrr}
		\toprule \textbf{\textit{Issue}} & \textbf{Estimado} & \textbf{Final}\\
		\toprule
		Añadir pantalla de <<about>> & 5 & 5 \\
		Comentar código & 5 & 5 \\
		Generar javadoc & 3 & 1 \\
		Terminar los anexos & 21 & 21 \\
		Generar test con Espresso & 5 & 5 \\
		Mejorar el README & 1 & 1 \\
		\bottomrule
	\end{tabularx}
	\caption{Tareas del sprint 26}
\end{table}

\subsection{Coste total de cada sprint}

En la siguiente tabla se muestran los costes estimados y finales totales de cada sprint y la suma del coste final del proyecto. 

\begin{table}[H]
	\begin{tabularx}{\textwidth}{Xrr}
		\toprule \textbf{\textit{Sprint}} & \textbf{Estimado} & \textbf{Final}\\
		\toprule
		Sprint 1 & 15 & 15 \\
		Sprint 2 & 22 & 22 \\
		Sprint 3 & 21 & 26 \\
		Sprint 4 & 21 & 34 \\
		Sprint 5 & 26 & 29 \\
		Sprint 6 & 31 & 31 \\
		Sprint 7 & 24 & 16 \\
		Sprint 8 & 37 & 39 \\
		Sprint 9 & 13 & 16 \\
		Sprint 10 & 8 & 16 \\
		Sprint 11 & 13 & 13 \\
		Sprint 12 & 18 & 18 \\
		Sprint 13 & 23 & 15 \\
		Sprint 14 & 26 & 34 \\
		Sprint 15 & 21 & 21 \\
		Sprint 16 & 20 & 28 \\
		Sprint 17 & 16 & 16 \\
		Sprint 18 & 21 & 29 \\
		Sprint 19 & 11 & 11 \\
		Sprint 20 & 26 & 31 \\
		Sprint 21 & 34 & 39 \\
		Sprint 22 & 8 & 8 \\
		Sprint 23 & 12 & 12 \\
		Sprint 24 & 21 & 21 \\
		Sprint 25 & 21 & 30 \\
		Sprint 26 & 40 & 38 \\
		\midrule
		\textbf{Total} & 549 & 608 \\
		\bottomrule
	\end{tabularx}
	\caption{Coste de cada sprint.}
\end{table}

\section{Estudio de viabilidad}

\subsection{Viabilidad económica}

Para considerar la viabilidad económica del proyecto se deben calcular los costes derivados de su realización. Se van a tener en cuenta tanto el coste del personal como el del softwares y el hardware empleados. Dado que este proyecto se ha realizado de forma conjunta con José Luis Garrido Labrador, ambos calcularemos el coste asumiendo que el proyecto cuenta con dos empleados.

\subsubsection{Costes de personal} 

Siguiendo estas consideraciones calculamos el coste total de personal de la siguiente forma: 

\begin{table}[H]
	\centering
	\begin{tabular}[]{@{}l r@{}}
		\toprule
		\textbf{Concepto} & \textbf{Coste(\euro{})} \\
		\midrule
		Salario mensual neto~\cite{salariales} & 1.225,7 \\
		Retención IRPF (15\%) & 216,3 \\
		Seguridad Social (28,3\%) & 569,16 \\
		Salario mensual bruto & 2.011,16 \\\hubu
		\textbf{Total 7 meses y dos empleados} &  28.156,24 \\
		\bottomrule
	\end{tabular}
	\caption{Costes de personal.}
	\label{tab:costes_personal}
\end{table}


\subsubsection{Costes del \textit{software}}

El \textit{software} empleado no supone ningún coste en este proyecto ya que todas las herramientas y bibliotecas utilizadas son de código abierto o gratuitas. 

\subsubsection{Costes del \textit{hardware}}

Para el desarrollo de este proyecto no se ha adquirido ningún \textit{hardware} nuevo, por lo que únicamente se incluirán los costes del material con el que ya se contaba asumiendo una amortización en 5 años, y calculando solo el coste de amortización correspondiente a la duración del proyecto (7 meses): 

\begin{table}[H]
	\centering
	\begin{tabular}[]{@{}l c r@{}}
		\toprule
		\textbf{Concepto} & \textbf{Coste(\euro{})} & \textbf{Coste amortizado(\euro{})} \\
		\midrule
		Dispositivo móvil & 150 & 17,5 \\
		Ordenador portátil (x2) & 800 & 93,33 \\
		\textit{MainFrame} & 3.000 & 350 \\ 
		GPU (x3) & 4.500 & 525 \\\hubu
		\textbf{Total} & 8.450 & 985,83 \\
		\bottomrule
	\end{tabular}
	\caption{Costes de \textit{hardware}.}
	\label{tab:costes_hardware}.
\end{table}

\subsubsection{Coste total}

Teniendo en cuenta los costes de personal y de \textit{hardware}, el coste económico total del proyecto asciende a:

\begin{table}[H]
	\centering
	\begin{tabular}[]{@{}l r@{}}
		\toprule
		\textbf{Concepto} & \textbf{Coste(\euro{})} \\
		\midrule
		Coste de personal & 28.156,24 \\ 
		Coste del \textit{hardware} & 985,83 \\\hubu
		\textbf{Total} & \textbf{29.142,07} \\	
		\bottomrule	
	\end{tabular}
	\caption{Coste total.}
	\label{tab:coste_total}
\end{table}

\subsection{Viabilidad legal}

En esta sección se hablará de la viabilidad legal del proyecto en términos de la licencia del software generado. A la hora de escoger la licencia más adecuada para nuestro software estamos limitados por las condiciones de las licencias de las herramientas o bibliotecas que hemos empleado para generarlos. 

En la aplicación se han usado herramientas y bibliotecas con las siguientes licencias ordenadas de más a menos permisivas: 

\begin{itemize}
	\item MIT: Socket.IO-client Library. 
	\item Apache 2.0: Android Studio, Gradle, Android Support Library, MPAndroidChart. 
\end{itemize}

Para la fase de investigación y documentación se usan herramientas y bibliotecas con las siguientes licencias ordenadas de más a menos permisivas: 

\begin{itemize}
	\item MIT: tsfresh.
	\item Zero clause BSD: tmux. 
	\item Modified BSD: Anaconda, Scikit-Learn, Jupyter Notebook.  
	\item GPLv2: Pencil. 
	\item GPLv3: Weka, DEAP. 
\end{itemize}

Todas las licencias son compatibles con GPL, siendo GPLv3 la más restrictiva. Para este proyecto se ha decidido usar la licencia AGPLv3\cite{wiki:agpl} (\textit{GNU Affero General Public License v3}), derivada de GPLv3 pero diseñada específicamente para asegurar la cooperación con la comunidad en el caso de software que corra en servidores de red.





