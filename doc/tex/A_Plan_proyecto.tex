\apendice{Plan de Proyecto Software}

\section{Introducción}

\section{Planificación temporal}

La planificación temporal se ha realizado adaptando la metodología \textit{Scrum} a un proyecto educativo, con los cambios que esto conlleva.

\begin{itemize}
	\item El desarrollo se ha basado en iteraciones o \textit{sprints} de una semana de duración aproximadamente.
	\item  Cada uno de los \textit{sprints} contiene las tareas o \textit{issues} que se realizaron esa semana. 
	\item Cada tarea tiene asociado un coste, que simboliza su dificultad en cuanto al esfuerzo que se estima invertir en ella.
	\item En caso de que la estimación del coste resultara inexacta al realizar el \textit{issue}, este se modificó para reflejar el esfuerzo real empleado.
	\item Al finalizar cada \textit{sprint} se realizaba una reunión de revisión con los tutores donde se exponían los progresos realizados y se planificaba el siguiente \textit{sprint}.
\end{itemize}

\subsection{Sprint 1}

Fecha: 19/12/1018 - 23/12/2018

El primer \textit{sprint} consistió en realizar una exploración bibliográfica inicial sobre el estado del arte. 

\begin{table}[h]
	 \begin{tabularx}{\linewidth}{|Xrr|}
	 	\hline \textbf{\textit{Issue}} & \textbf{Estimado} & \textbf{Final}\\
	 	\hline Crear y configurar repositorio & 2 & 2 \\
	 	\hline Exploración bibliográfica inicial & 13 & 13 \\
	 	\hline
	 \end{tabularx}
\end{table}

\subsection{Sprint 2}

Fecha: 23/12/2018 - 29/12/2018

Se continuó la exploración bibliográfica inicial, centrándose en artículos especialmente interesantes encontrados hasta el momento y se comenzó la exploración bibliográfica sobre otros métodos aplicables al problema. 

\begin{table}[h]
	\begin{tabularx}{\textwidth}{|Xrr|}
		\hline \textbf{\textit{Issue}} & \textbf{Estimado} & \textbf{Final}\\
		\hline Continuación de la exploración bibliográfica inicial & 8 & 8 \\
		\hline Exploración bibliográfica sobre otros métodos aplicables al problema & 8 & 8 \\
		\hline Lectura de "Automated Epileptic Seizure Detection Methods: A Review Study" & 8 & 8 \\
		\hline Instalar y configurar cliente VPN & 2 & 2 \\
		\hline
	\end{tabularx}
\end{table}

\subsection{Sprint 3}

Fecha: 29/12/2018 - 11/01/2019

Se inició la documentación y se empezó a trabajar en la visualización de los datos en bruto y de algunos datos estadísticos.

\begin{table}[h]
	\begin{tabularx}{\textwidth}{|Xrr|}
		\hline \textbf{\textit{Issue}} & \textbf{Estimado} & \textbf{Final}\\
		\hline Iniciar documentación &  &  \\
		\hline Instalar entorno y librerías de Python & 5 & 5 \\
		\hline Aprender a usar librerías & 8 & 8 \\
		\hline Procesar y mostrar datos & 8 & 8 \\
		\hline
	\end{tabularx}
\end{table}

\subsection{Sprint 4}

Fecha: 11/01/2019 - 18/01/2019

Se configuró el acceso al computador del departamento para probar técnicas de reducción de la dimensionalidad de los datos y algunas opciones básicas de filtrado y suavizado de la señal. 

\begin{table}[h]
	\begin{tabularx}{\textwidth}{|Xrr|}
		\hline \textbf{\textit{Issue}} & \textbf{Estimado} & \textbf{Final}\\
		\hline Configurar acceso a gamma & 5 & 5 \\
		\hline Probar opciones filtrado y suavizado & 8 & 8 \\
		\hline Probar otras formas de proyección de datos & 8 & 21 \\
		\hline
	\end{tabularx}
\end{table}

\subsection{Sprint 5}

Fecha: 18/01/2019 - 25/01/2019

Se hicieron cambios en el preprocesado, se probaron otras formas de filtrado de la señal y se estudiaron los puntos clave de las proyecciones del \textit{sprint} anterior. 

\begin{table}[h]
	\begin{tabularx}{\textwidth}{|Xrr|}
		\hline \textbf{\textit{Issue}} & \textbf{Estimado} & \textbf{Final}\\
		\hline Leer apuntes de minería de datos & 8 & 8 \\
		\hline Modificar preprocesado & 3 & 3 \\
		\hline Representar señales en torno a la crisis epiléptica & 5 & 5 \\
		\hline Probar formas de filtrado de la señal & 5 & 5 \\
		\hline Estudiar los puntos clave de las proyecciones & 5 & 8 \\
		\hline
	\end{tabularx}
\end{table}

\subsection{Sprint 6}

Fecha: 25/01/2019 - 31/02/2019

Se centraron las pruebas en las proyecciones con mejor rendimiento, concretamente en MDS, y se iniciaron la documentación de la planificación temporal y el cuaderno de investigación. 

\begin{table}[h]
	\begin{tabularx}{\textwidth}{|Xrr|}
		\hline \textbf{\textit{Issue}} & \textbf{Estimado} & \textbf{Final}\\
		\hline Cambiar a proyecciones con mejor rendimiento & 13 & 13 \\
		\hline Pasar cálculos estadísticos a funciones & 5 & 5 \\
		\hline Documentar 5 primeros Sprints en el Plan de Proyecto & 8 & 8 \\
		\hline Documentar investigación en overleaf & 5 & 5 \\
		\hline
	\end{tabularx}
\end{table}

\section{Estudio de viabilidad}

\subsection{Viabilidad económica}

\subsection{Viabilidad legal}


