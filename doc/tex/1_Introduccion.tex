\capitulo{1}{Introducción}

La epilepsia es un trastorno neurológico provocado por la alteración de la actividad normal de una región cerebral, que desencadena crisis caracterizadas por convulsiones musculares reiteradas y, en ocasiones, pérdida de la consciencia. Se trata de una de las enfermedades neurológicas más habituales, y aunque las crisis epilépticos pueden experimentarse de forma aislada o durante periodos de tiempo limitados, una gran cantidad de la población las sufre de forma crónica. Según la Federación Española de Epilepsia~\cite{fed_esp_epilepsia} alrededor de 700.000 personas padecen o han padecido epilepsia a lo largo de su vida, y más de 200.000 la padecen de forma activa. 

Para una persona con epilepsia crónica, la detección inmediata de una crisis es vital para permitir la aplicación de primeros los auxilios que ayuden a evitar consecuencias permanentes. Actualmente, en la bibliografía se habla de varias técnicas para la detección automática de crisis~\cite{ramgopal2014epilepsy,tzallas2012automated}, la mayoría basadas en Electroencefalogramas (EEG) o en el uso de dispositivos portátiles (\textit{wearables}) como pulseras inteligentes basadas en la monitorización de las constantes vitales del paciente. 

La detección mediante EEG se usa principalmente para diagnósticos médicos, ya que los dispositivos que se necesitan son demasiado costosos o aparatosos como para ser usados en el día a día del paciente. Por otro lado, las pulseras inteligentes resultan más convenientes para este cometido, ya que son más baratas y cómodas de utilizar. Sin embargo, ambas técnicas requieren del uso consciente y continuado de los dispositivos de detección, lo que puede suponer un inconveniente para pacientes dependientes o con necesidades especiales. Por esta razón se propone el uso de colchones inteligentes para la detección automática de crisis nocturnas, mediante sensores de presión y biométricos incorporados en el interior del propio colchón. 

Sea cual sea el dispositivo utilizado, la captación de los datos (actividad eléctrica del cerebro, constantes vitales, presiones, etc.) no basta para detectar una crisis. Es necesario un procesado adecuado para determinar si estos datos corresponden o no con una crisis epiléptica. Para ello, las técnicas de minería de datos permiten generar modelos de clasificación capaces de realizar esta tarea. Para este trabajo de fin de grado, el principal objetivo será encontrar un modelo de clasificación efectivo mediante la aplicación de este tipo de técnicas sobre los datos disponibles.

\section{Estructura de la memoria}

Esta memoria incluye los siguientes apartados: 

\begin{itemize}
	\item \textbf{Objetivos del proyecto:} se definen los objetivos generales, técnicos y personales que se persiguen con la realización de este trabajo. 
	\item \textbf{Conceptos teóricos:} TODO
	\item \textbf{Técnicas y herramientas:} TODO
	\item \textbf{Aspectos relevantes del desarrollo del proyecto:} TODO
	\item \textbf{Trabajo relacionados:} TODO
	\item \textbf{Conclusiones y líneas de trabajo futuras:} TODO
\end{itemize}

\section{Materiales adjuntos}

\begin{itemize}	
	\item \textbf{Anexos:}
	\begin{itemize}
		\item Plan de Proyecto Software 
		\item Especificación de Requisitos
		\item Especificación de diseño
		\item Documentación técnica de programación 
		\item Documentación de usuario 
	\end{itemize}
	\item \textbf{Cuaderno de investigación:} recoge las explicaciones de todas las técnicas probadas, los resultados y las comparativas de todos los experimentos realizados con el fin de encontrar el mejor modelo de clasificación para el problema. 
	\item \textbf{Experimentos:} Conjunto de notebooks de jupyter que contienen todos los experimentos realizados. 
	\item \textbf{App de Android:} Archivo .apk para la distribución e instalación de la aplicación para Android desarrollada con el fin de mostrar la aplicabilidad del modelo de clasificación. 
\end{itemize}

A la memoria y a todos los demás materiales adjuntos se puede acceder a través del \href{https://github.com/aog0036/TFG-SmartBeds}{repositorio}~\cite{repo_smartbeds} del proyecto en GitHub. 


