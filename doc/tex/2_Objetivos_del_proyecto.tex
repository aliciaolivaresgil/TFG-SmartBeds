\capitulo{2}{Objetivos del proyecto}

En este apartado se exponen los objetivos perseguidos con la realización de este trabajo: 

\section{Objetivos generales}

\begin{itemize}
	\item Investigar sobre técnicas del estado del arte aplicadas a problemas similares. 
	\item Aplicar técnicas de minería de datos siguiendo los pasos del Descubrimiento de Conocimiento en Bases de Datos (KDD).
	\item Explorar, aplicar y comparar distintas formas de preprocesado de los datos (filtrado, normalización, transformación, etc.). 
	\item Usar técnicas de proyección de datos a 2 dimensiones para comprobar si los instancias de ``crisis'' son fácilmente separables de las instancias de ``no crisis''. 
	\item Probar modelos de clasificación para conjuntos de datos con preprocesados basados en estadísticas simples.
	\item Probar modelos de clasificación para conjuntos de datos con preprocesados basados en características de series temporales.  
	\item Probar modelos de clasificación mediante \textit{ensembles} para conjuntos de datos desequilibrados. 
	\item Probar detección de anomalías mediante un modelo \textit{One-Class}.
	\item Comparar el rendimiento de los modelos obtenidos. 
	\item Comparar distintas métricas usadas para evaluar el rendimiento los modelos obtenidos. 
	\item Generar un modelo de clasificación capaz de detectar crisis epilépticas a partir de los datos disponibles. 
	\item Desarrollar una app de Android para mostrar la aplicabilidad del modelo de clasificación generado. 
\end{itemize}

\section{Objetivos técnicos}

\begin{itemize}
	\item Usar Overleaf como herramienta de edición online de LaTeX para la generación del cuaderno de investigación conjunto.  
	\item Realizar y exponer los resultados de los experimentos en notebooks de jupyter empleando código python. 
	\item Usar herramientas de sklearn para la proyección de los datos y para la obtención y evaluación de los modelos.
	\item Generar un conjunto de Transformadores compatibles con sklearn para poder aplicar las técnicas de preprocesado de forma directa. 
	\item Visualizar los datos y los resultados de los experimentos con pandas y matplotlib. 
	\item Usar la librería tsfresh para extracción de características en series temporales. 
	\item Usar la herramienta Weka para probar modelos de clasificación mediante ensembles en conjuntos de datos desequilibrados. 
	\item Desarrollar una app Android con soporte para API 23 y superiores. 
	\item Usar peticiones HTTP desde la app para la comunicación con la API del servidor remoto. 
	\item Usar Socket.io desde la app para la obtención de datos en tiempo real del servidor remoto.
	\item Usar la plataforma GitHub junto con la extensión ZenHub para la gestión del proyecto. 
	\item Aplicar la metodología Scrum adaptada a un proyecto con fines educativos. 
	\item Realizar test TODO. 
	
	
\end{itemize}

\section{Objetivos personales}

\begin{itemize}
	\item Iniciarme en el campo de la investigación. 
	\item Explorar técnicas y herramientas aplicadas a la minería de datos. 
	\item Aprender a generar documentación con LaTeX. 
	\item Iniciarme en el desarrollo de aplicaciones Android. 
\end{itemize}
