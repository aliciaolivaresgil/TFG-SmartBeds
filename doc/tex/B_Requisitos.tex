\apendice{Especificación de Requisitos}

\section{Introducción}

Aunque este trabajo se centra sobre todo en investigación, cuando hablamos de requisitos nos referiremos a los de la aplicación Android generada. En esta sección se enumerarán los requisitos funcionales y no funcionales de la aplicación, y se definirán los casos de uso derivados.  

\section{Objetivos generales}

En la memoria se exponen los objetivos generales del trabajo, los cuales, dada la naturaleza del trabajo, se centran principalmente en la fase de investigación. En este apartado nos centraremos en los requisitos relativos al último de los objetivos generales expuestos: 

<<\textit{Desarrollar una app de Android para mostrar la aplicabilidad del modelo de clasificación generado.}>>

\section{Catalogo de requisitos}

Aquí se enumeran los requisitos funcionales y no funcionales de la aplicación desarrollada para dispositivos Android. Dado que mi compañero de proyecto José Luis Garrido Labrador y yo hemos realizado dos aplicaciones (una web y una para Android) con el mismo objetivo y las mismas funcionalidades, la especificación de los requisitos  se ha realizado de forma conjunta, y por lo tanto, muchos de los puntos de estos apartados coincidirán en ambos trabajos. 

\subsection{Requisitos funcionales}
 
\begin{itemize}
	\item \textbf{RF-1 Confidencialidad del sistema:} Solamente los usuarios autorizados podrán acceder al sistema. 
	
\begin{itemize}
		\item \textbf{RF-1.1 Identificación de usuario:} los usuarios se identificarán con un \textit{nickname} y una contraseña 
		\item \textbf{RF-1.2 Rol de administración:} existirá un usuario especial que podrá administrar el sistema completamente sin restricciones.
		\item \textbf{RF-1.3 Visualización de una cama:} los usuarios validados deben poder observar los datos en tiempo real de las camas disponibles. 
		\item \textbf{RF-1.4 Restricción de acceso:} los usuarios solamente podrán tener acceso a los datos de las camas permitidas. 
		\item \textbf{RF-1.5 Acceso completo al administrador:} el administrador debe poder acceder a los datos de todas las camas existentes.
\end{itemize}
	
	\item \textbf{RF-2 Gestión de las camas:} El administrador debe poder gestionar las camas pudiendo añadir, modificar, borrar y dar acceso a un usuario a los datos de una cama determinada. 
	
\begin{itemize}
		\item \textbf{RF-2.1 Añadir cama:} el administrador debe poder añadir una nueva cama al sistema.
		\item \textbf{RF-2.2 Modificar cama:} el administrador debe poder modificar los datos una cama existente.
		\item \textbf{RF-2.3 Borrar cama:} el administrador debe poder borrar una cama del sistema.
		\item \textbf{RF-2.4 Asignar camas a usuarios:} el administrador se encarga de decidir qué usuario puede acceder a los datos de qué cama.
\end{itemize}
	
	\item \textbf{RF-3 Gestión de los usuarios:} el administrador debe poder gestionar los usuarios pudiendo añadir, modificar y borrar. El usuario debe poder gestionar su propia contraseña. 

\begin{itemize}
		\item \textbf{RF-3.1 Añadir usuario:} el administrador debe poder añadir un nuevo usuario al sistema.
		\item \textbf{RF-3.2 Modificar usuario:} el administrador debe poder modificar los datos un usuario existente. Igualmente el usuario debe poder modificar su propia contraseña. 
		\item \textbf{RF-3.3 Borrar usuario:} el administrador debe poder borrar un usuario del sistema.
\end{itemize}

	\item \textbf{RF-4 Visualización de los datos:} los usuarios deben poder ver, de las camas disponibles, el estado actual del paciente, la probabilidad de crisis epiléptica, sus constantes vitales y las presiones. 
	
\end{itemize}

\subsection{Requisitos no funcionales}

\begin{itemize}
	\item \textbf{RNF-1 Usabilidad:} la aplicación debe cumplir estándares de usabilidad teniendo una curva de aprendizaje baja y un uso de metáforas adecuado.
	\item \textbf{RNF-2 Confidencialidad:} los datos de las camas, al ser en parte constantes vitales de pacientes, solamente han de ser accesibles por los usuarios permitidos.
	\item \textbf{RNF-3 Escalabilidad:} el sistema debe ser escalable para adaptarse de manera correcta a un incremento de carga del sistema.
	\item \textbf{RNF-4 Seguridad:} los usuarios deben poder identificarse sólidamente con el sistema sin que sus datos o sus credenciales (\textit{tokens}) sean accesibles por terceros, incluso el administrador.
\end{itemize}


\section{Especificación de requisitos}

De la misma forma, en lo relativo a las funcionalidades del cliente, la especificación de los casos de uso se ha hecho de forma conjunta con mi compañero José Luis Garrido Labrador, por lo que los contenidos de este apartado coincidirán en gran medida con los suyos. 

\subsection{Actores}

En los casos de uso se distinguen dos actores: 
\begin{itemize}
	\item \textbf{Administrador:} Tiene acceso a la gestión de usuarios, la gestión de camas y la visualización de los datos de todas las camas existentes. 
	\item \textbf{Usuario:} Tiene acceso a la visualización de los datos de las camas que tiene asignadas y a la gestión de su propio usuario. 
\end{itemize}

\subsection{Diagramas de casos de uso}

\begin{figure}[H]
	\centering
	\includegraphics[width=1\textwidth]{../img/cu-n0.png}
	\caption{Diagrama de casos de uso, nivel 0.}
	\label{fig:cu-n0}
\end{figure}

\begin{figure}[H]
	\centering
	\includegraphics[width=1\textwidth]{../img/cu-n1-2.png}
	\caption{Diagrama de casos de uso, visualización de datos.}
	\label{fig:cu-n1.2}
\end{figure}

\begin{figure}[H]
	\centering
	\includegraphics[width=1\textwidth]{../img/cu-n1-3.png}
	\caption{Diagrama de casos de uso, administración de usuarios.}
	\label{fig:cu-n1.3}
\end{figure}

\begin{figure}[H]
	\centering
	\includegraphics[width=1\textwidth]{../img/cu-n1-4.png}
	\caption{Diagrama de casos de uso, administración de camas.}
	\label{fig:cu-n1.4}
\end{figure}

\subsection{Especificación de casos de uso}

\tablaSmallSinColores{Caso de uso 1: Iniciar sesión }{p{3cm} p{.75cm} p{9.5cm}}{tablaCU1}{
	\multicolumn{3}{p{10.25cm}}{CU-1: Iniciar sesión} \\
}
{
	Descripción                            & \multicolumn{2}{p{10.25cm}}{El usuario se identifica en el sistema} \\\hubu
	Precondiciones                         & \multicolumn{2}{p{10.25cm}}{No existe una sesión activa válida} \\\hubu
	Requisitos                         	   & \multicolumn{2}{p{10.25cm}}{RF-1, RF-1.1} \\\hubu
	Usuario                         	   & \multicolumn{2}{p{10.25cm}}{Anónimo} \\\hubu
	\multirow{3}{3.5cm}{Secuencia normal}  & Paso & Acción \\\cline{2-3}
	& 1    & El cliente envía sus credenciales al servidor \\\cline{2-3}
	& 2    & El servidor acepta las credenciales devolviendo el token de sesión \\\hubu
	Postcondiciones                        & \multicolumn{2}{p{10.25cm}}{El usuario tiene una sesión activa válida} \\\hubu
	\multirow{2}{3.5cm}{Excepciones}       & Paso & Acción \\\cline{2-3}
	& 2    & Si las credenciales son incorrectas el servidor responde con error \\\hubu
	Frecuencia                             & Alta \\\hubu
	Importancia                            & Crítico \\\hubu
	Comentarios                            & \multicolumn{2}{p{10.25cm}}{Es siempre lo primero que aparecerá} \\
}

\tablaSmallSinColores{Caso de uso 2: Visualizar de datos }{p{3cm} p{.75cm} p{9.5cm}}{tablaCU2}{
	\multicolumn{3}{p{10.25cm}}{CU-2: Visualizar de datos} \\
}
{
	Descripción                            & \multicolumn{2}{p{10.25cm}}{Ver lista de las camas disponibles} \\\hubu
	Precondiciones                         & \multicolumn{2}{p{10.25cm}}{Sesión activa válida} \\\hubu
	Requisitos                         	   & \multicolumn{2}{p{10.25cm}}{RF-1.3, RF-1.4} \\\hubu
	Usuario                         	   & \multicolumn{2}{p{10.25cm}}{Administrador y Usuario} \\\hubu
	\multirow{3}{3.5cm}{Secuencia normal}  & Paso & Acción \\\cline{2-3}
	& 1    & El cliente solicita ver las camas disponibles \\\hubu
	Postcondiciones                        & \multicolumn{2}{p{10.25cm}}{El cliente está en la pantalla de camas disponibles} \\\hubu
	Frecuencia                             & Alta \\\hubu
	Importancia                            & Alta \\
}

\tablaSmallSinColores{Caso de uso 2.1: Elegir cama }{p{3cm} p{.75cm} p{9.5cm}}{tablaCU21}{
	\multicolumn{3}{p{10.25cm}}{CU-2.1: Elegir cama} \\
}
{
	Descripción                            & \multicolumn{2}{p{10.25cm}}{Elegir cama} \\\hubu
	Precondiciones                         & \multicolumn{2}{p{10.25cm}}{Sesión activa válida} \\\hubu
	Requisitos                         	   & \multicolumn{2}{p{10.25cm}}{RF-1.3, RF-1.4, RF-4} \\\hubu
	Usuario                         	   & \multicolumn{2}{p{10.25cm}}{Logueado} \\\hubu
	\multirow{3}{3.5cm}{Secuencia normal}  & Paso & Acción \\\cline{2-3}
	& 1    & El cliente solicita ver las camas disponibles \\\cline{2-3}
	& 2    & El servidor abre conexiones paralelas para actualizar en tiempo real el estado de las camas \\\cline{2-3}
	& 3    & El cliente decide que cama ver \\\hubu
	Postcondiciones                        & \multicolumn{2}{p{10.25cm}}{El cliente entra en la ventana de los datos en tiempo real} \\\hubu
	Frecuencia                             & Alta \\\hubu
	Importancia                            & Alta \\
}

\tablaSmallSinColores{Caso de uso 2.2: Ver datos en tiempo real }{p{3cm} p{.75cm} p{9.5cm}}{tablaCU22}{
	\multicolumn{3}{p{10.25cm}}{CU-2.2: Ver datos en tiempo real} \\
}
{
	Descripción                            & \multicolumn{2}{p{10.25cm}}{Ver datos en tiempo real} \\\hubu
	Precondiciones                         & \multicolumn{2}{p{10.25cm}}{Sesión activa válida y cama existente y accesible} \\\hubu
	Requisitos                         	   & \multicolumn{2}{p{10.25cm}}{RF-1.3, RF-1.4, RF-4} \\\hubu
	Usuario                         	   & \multicolumn{2}{p{10.25cm}}{Administrador y usuario} \\\hubu
	\multirow{3}{3.5cm}{Secuencia normal}  & Paso & Acción \\\cline{2-3}
	& 1    & El cliente solicita una nueva conexión \\\cline{2-3}
	& 2    & El servidor provee una conexión en tiempo real con los datos \\\hubu
	Postcondiciones                        & \multicolumn{2}{p{10.25cm}}{El usuario tiene una conexión paralela abierta con los datos en tiempo real} \\\hubu
	\multirow{2}{3.5cm}{Excepciones}       & Paso & Acción \\\cline{2-3}
	& 2    & Si un paquete faltase o la señal fuera, débil se alertaría al usuario \\\hubu
	Frecuencia                             & Alta \\\hubu
	Importancia                            & Máxima \\}

\tablaSmallSinColores{Caso de uso 3: Administrar de usuarios }{p{3cm} p{.75cm} p{9.5cm}}{tablaCU3}{
	\multicolumn{3}{p{10.25cm}}{CU-3: Administrar de usuarios} \\
}
{
	Descripción                            & \multicolumn{2}{p{10.25cm}}{Administración de usuario: alta, baja y modificación} \\\hubu
	Precondiciones                         & \multicolumn{2}{p{10.25cm}}{Sesión de administrador válida} \\\hubu
	Requisitos                         	   & \multicolumn{2}{p{10.25cm}}{RF-3} \\\hubu
	Usuario                         	   & \multicolumn{2}{p{10.25cm}}{Administrador} \\\hubu
	\multirow{3}{3.5cm}{Secuencia normal}  & Paso & Acción \\\cline{2-3}
	& 1    & El administrador entra en el menú de administración de usuarios \\\hubu
	Postcondiciones                        & \multicolumn{2}{p{10.25cm}}{El administrador está en el menú de administración de usuarios} \\\hubu
	Frecuencia                             & Baja \\\hubu
	Importancia                            & Alta \\
}

\tablaSmallSinColores{Caso de uso 3.1: Añadir usuarios }{p{3cm} p{.75cm} p{9.5cm}}{tablaCU31}{
	\multicolumn{3}{p{10.25cm}}{CU-3.1: Añadir usuarios} \\
}
{
	Descripción                            & \multicolumn{2}{p{10.25cm}}{Añadir usuarios} \\\hubu
	Precondiciones                         & \multicolumn{2}{p{10.25cm}}{Sesión de administración activa} \\\hubu
	Requisitos                         	   & \multicolumn{2}{p{10.25cm}}{RF-3.1} \\\hubu
	Usuario                         	   & \multicolumn{2}{p{10.25cm}}{Administrador} \\\hubu
	\multirow{3}{3.5cm}{Secuencia normal}  & Paso & Acción \\\cline{2-3}
	& 1    & El administrador elige añadir un nuevo usuario \\\cline{2-3}
	& 2    & Se introduce un nombre de usuario para identificarlo \\\cline{2-3}
	& 3    & Se introduce una contraseña dos veces \\\cline{2-3}
	& 4    & Se almacenan los datos \\\hubu
	Postcondiciones                        & \multicolumn{2}{p{10.25cm}}{Existe un nuevo usuario en el sistema} \\\hubu
	\multirow{2}{3.5cm}{Excepciones}       & Paso & Acción \\\cline{2-3}
	& 2    & Si el nickname existiese \\\cline{2-3}
	& 3    & La contraseña añadida no coincide en las dos ocasiones \\\hubu
	Frecuencia                             & Baja \\\hubu
	Importancia                            & Alta \\
}

\tablaSmallSinColores{Caso de uso 3.2: Modificar contraseña }{p{3cm} p{.75cm} p{9.5cm}}{tablaCU32}{
	\multicolumn{3}{p{10.25cm}}{CU-3.2: Modificar contraseña} \\
}
{
	Descripción                            & \multicolumn{2}{p{10.25cm}}{Cambiar la contraseña de un usuario} \\\hubu
	Precondiciones                         & \multicolumn{2}{p{10.25cm}}{Sesión activa válida, usuario existente} \\\hubu
	Requisitos                         	   & \multicolumn{2}{p{10.25cm}}{RF-3.2} \\\hubu
	Usuario                         	   & \multicolumn{2}{p{10.25cm}}{Administrador y Usuario} \\\hubu
	\multirow{3}{3.5cm}{Secuencia normal}  & Paso & Acción \\\cline{2-3}
	& 1    & Si es usuario normal ir a 3 \\\cline{2-3}
	& 2    & Si es administrador elegir a qué usuario cambiar la contraseña \\\cline{2-3}
	& 3    & Se introduce una contraseña nueva dos veces \\\cline{2-3}
	& 4    & Se actualizan los datos \\\hubu
	Postcondiciones                        & \multicolumn{2}{p{10.25cm}}{La contraseña ha cambiado} \\\hubu
	\multirow{2}{3.5cm}{Excepciones}       & Paso & Acción \\\cline{2-3}
	& 3    & La contraseña añadida no coincide en las dos ocasiones \\\hubu
	Frecuencia                             & Baja \\\hubu
	Importancia                            & Alta \\
}

\tablaSmallSinColores{Caso de uso 3.3: Borrar usuario }{p{3cm} p{.75cm} p{9.5cm}}{tablaCU33}{
	\multicolumn{3}{p{10.25cm}}{CU-3.3: Borrar usuario} \\
}
{
	Descripción                            & \multicolumn{2}{p{10.25cm}}{Elimina un usuario de la base de datos} \\\hubu
	Precondiciones                         & \multicolumn{2}{p{10.25cm}}{Sesión de administración válida, usuario existente} \\\hubu
	Requisitos                         	   & \multicolumn{2}{p{10.25cm}}{RF-3.3} \\\hubu
	Usuario                         	   & \multicolumn{2}{p{10.25cm}}{Administrador} \\\hubu
	\multirow{3}{3.5cm}{Secuencia normal}  & Paso & Acción \\\cline{2-3}
	& 1    & Elegir a que usuario (no administrador) eliminar \\\cline{2-3}
	& 2    & Eliminar usuario y todos los datos vinculados \\\hubu
	Postcondiciones                        & \multicolumn{2}{p{10.25cm}}{El usuario ha sido eliminado} \\\hubu
	Frecuencia                             & Baja \\\hubu
	Importancia                            & Media \\
}

\tablaSmallSinColores{Caso de uso 4: Administrar de camas }{p{3cm} p{.75cm} p{9.5cm}}{tablaCU4}{
	\multicolumn{3}{p{10.25cm}}{CU-4: Administrar de camas} \\
}
{
	Descripción                            & \multicolumn{2}{p{10.25cm}}{Administración de camas: alta, baja, modificación y asignación a usuarios} \\\hubu
	Precondiciones                         & \multicolumn{2}{p{10.25cm}}{Sesión de administración válida} \\\hubu
	Requisitos                         	   & \multicolumn{2}{p{10.25cm}}{RF-2} \\\hubu
	Usuario                         	   & \multicolumn{2}{p{10.25cm}}{Administrador} \\\hubu
	\multirow{3}{3.5cm}{Secuencia normal}  & Paso & Acción \\\cline{2-3}
	& 1    & El administrador entra en el menú de administración de camas \\\hubu
	Postcondiciones                        & \multicolumn{2}{p{10.25cm}}{El administrador está en el menú de administración de camas} \\\hubu
	Frecuencia                             & Baja \\\hubu
	Importancia                            & Media \\
}

\tablaSmallSinColores{Caso de uso 4.1: Añadir cama }{p{3cm} p{.75cm} p{9.5cm}}{tablaCU41}{
	\multicolumn{3}{p{10.25cm}}{CU-4.1: Añadir cama} \\
}
{
	Descripción                            & \multicolumn{2}{p{10.25cm}}{Añadir cama} \\\hubu
	Precondiciones                         & \multicolumn{2}{p{10.25cm}}{Sesión de administración válida} \\\hubu
	Requisitos                         	   & \multicolumn{2}{p{10.25cm}}{RF-2.1} \\\hubu
	Usuario                         	   & \multicolumn{2}{p{10.25cm}}{Administrador} \\\hubu
	\multirow{3}{3.5cm}{Secuencia normal}  & Paso & Acción \\\cline{2-3}
	& 1    & El administrador elige añadir una nueva cama \\\cline{2-3}
	& 2    & Se introduce el grupo multicast de la cama (IP y Puerto) \\\cline{2-3}
	& 3    & Se introduce el nombre identificador\\\cline{2-3}
	& 4    & Se almacenan los datos \\\hubu
	Postcondiciones                        & \multicolumn{2}{p{10.25cm}}{Existe una nueva cama en el sistema} \\\hubu
	\multirow{2}{3.5cm}{Excepciones}       & Paso & Acción \\\cline{2-3}
	& 2    & El grupo multicast pertenece a otra cama \\\cline{2-3}
	& 3    & El nombre identificativo existe para otra cama \\\hubu
	Frecuencia                             & Media \\\hubu
	Importancia                            & Crítica \\\hubu
	Comentarios                            & \multicolumn{2}{p{10.25cm}}{El grupo multicast se configura en la cama y el administrador solamente debe conocerlo, no configurar la cama física} \\
}

\tablaSmallSinColores{Caso de uso 4.2: Modificar cama }{p{3cm} p{.75cm} p{9.5cm}}{tablaCU42}{
	\multicolumn{3}{p{10.25cm}}{CU-2.2: Modificar cama} \\
}
{
	Descripción                            & \multicolumn{2}{p{10.25cm}}{Modificar los datos de la cama} \\\hubu
	Precondiciones                         & \multicolumn{2}{p{10.25cm}}{Sesión de administración válida, cama existente} \\\hubu
	Requisitos                         	   & \multicolumn{2}{p{10.25cm}}{RF-2.2} \\\hubu
	Usuario                         	   & \multicolumn{2}{p{10.25cm}}{Administrador} \\\hubu
	\multirow{3}{3.5cm}{Secuencia normal}  & Paso & Acción \\\cline{2-3}
	& 1    & Se elige que cama modificar \\\cline{2-3}
	& 2    & Se actualizan los datos a conveniencia del administrador según CU-4.1\\\cline{2-3}
	& 4    & Se actualizan los datos \\\hubu
	Postcondiciones                        & \multicolumn{2}{p{10.25cm}}{Los datos de la cama se modifican} \\\hubu
	\multirow{2}{3.5cm}{Excepciones}       & Paso & Acción \\\cline{2-3}
	& 2    & Mismas excepciones que en CU-4.1 \\\hubu
	Frecuencia                             & Baja \\\hubu
	Importancia                            & Alta \\
}

\tablaSmallSinColores{Caso de uso 4.3: Borrar cama }{p{3cm} p{.75cm} p{9.5cm}}{tablaCU43}{
	\multicolumn{3}{p{10.25cm}}{CU-4.3: Borrar cama} \\
}
{
	Descripción                            & \multicolumn{2}{p{10.25cm}}{Elimina una cama de la base de datos} \\\hubu
	Precondiciones                         & \multicolumn{2}{p{10.25cm}}{Sesión de administrador válida, cama existente} \\\hubu
	Requisitos                         	   & \multicolumn{2}{p{10.25cm}}{RF-2.3} \\\hubu
	Usuario                         	   & \multicolumn{2}{p{10.25cm}}{Administrador} \\\hubu
	\multirow{3}{3.5cm}{Secuencia normal}  & Paso & Acción \\\cline{2-3}
	& 1    & Elegir a que cama eliminar \\\cline{2-3}
	& 2    & Eliminar cama y todos los datos vinculados \\\hubu
	Postcondiciones                        & \multicolumn{2}{p{10.25cm}}{La cama ya no está en la base de datos} \\\hubu
	Frecuencia                             & Baja \\\hubu
	Importancia                            & Media \\
}

\tablaSmallSinColores{Caso de uso 4.4: Asignar cama a usuario }{p{3cm} p{.75cm} p{9.5cm}}{tablaCU44}{
	\multicolumn{3}{p{10.25cm}}{CU-4.4: Asignar cama a usuario} \\
}
{
	Descripción                            & \multicolumn{2}{p{10.25cm}}{Permite a un usuario ver los datos de una cama o quitar ese permiso} \\\hubu
	Precondiciones                         & \multicolumn{2}{p{10.25cm}}{Sesión de administración válida, cama y usuario existentes} \\\hubu
	Requisitos                         	   & \multicolumn{2}{p{10.25cm}}{RF-2.4} \\\hubu
	Usuario                         	   & \multicolumn{2}{p{10.25cm}}{Administrador} \\\hubu
	\multirow{3}{3.5cm}{Secuencia normal}  & Paso & Acción \\\cline{2-3}
	& 1    & Elegir cama \\\cline{2-3}
	& 2    & Elegir usuario \\\cline{2-3}
	& 3    & Si la relación existe se puede eliminar el permiso \\\cline{2-3}
	& 3    & Si la relación no existe se puede crear el permiso \\\hubu
	Postcondiciones                        & \multicolumn{2}{p{10.25cm}}{El usuario tiene acceso a la cama, o pierde el mismo} \\\hubu
	Frecuencia                             & Media \\\hubu
	Importancia                            & Crítica \\
}



